\documentclass[11pt,a4paper]{article}
\usepackage[utf8]{inputenc}
\usepackage[T1]{fontenc}
\usepackage[polish]{babel}
\usepackage{amsmath}
\usepackage{amsfonts}
\usepackage{amssymb}
\author{Maksymilian Ryder}
\title{Złożone struktury danych}

\usepackage{hyperref}

\usepackage{booktabs}
\usepackage{multirow}

\usepackage[dvipsnames]{xcolor}

\usepackage{pgfplots}
\pgfplotsset{compat=newest}
\usepgfplotslibrary{units}
\pgfplotsset{
	every axis legend/.append style={
		at={(1.01,1)},
		anchor=north west
	}
}

\begin{document}
\maketitle
\medskip 
\medskip 
\medskip 
\medskip 
Implementacja algorytmu konstruowania drzewa AVL metodą przeszukiwania binarnego oraz algorytmu konstruowania „losowego” drzewa BST


\section{Wykresy}
\subsection{Czas Tworzenia}
\medskip
\begin{tikzpicture}
\begin{axis}[
	width=10cm,
	height=12cm,
	ylabel=$time$,
	xlabel=$size$,
	y unit=s, y SI prefix=nano,
%	xmode=log,
%	ymode=log,
	title={Creation Time}
	]
\addplot [smooth, mark=*, mark size=0.5pt]
	plot [error bars/.cd, y dir = both, y explicit]
	table[
%		y error=creation_time_deviation,
		x=size,
		y=creation_time,
		col sep=semicolon
	]
	{data/RandomListGenerator/AVL.csv};
\addlegendentry{AVL}

\addplot [smooth, color=Dandelion, mark=*, mark size=0.5pt]
	plot [error bars/.cd, y dir = both, y explicit]
	table[
%		y error=creation_time_deviation,
		x=size,
		y=creation_time,
		col sep=semicolon
	]
	{data/RandomListGenerator/BST.csv};
\addlegendentry{BST}
\end{axis}
\end{tikzpicture}

\subsection{Czas Znajdywania najmniejszej wartości}

\begin{tikzpicture}
\begin{axis}[
	width=10cm,
	height=5cm,
	ylabel=$time$,
	xlabel=$size$,
	y unit=s, y SI prefix=nano,
	title={Trace Smallest Value Time}
	]
\addplot [smooth, mark=*, mark size=0.5pt]
	plot [error bars/.cd, y dir = both, y explicit]
	table[
		x=size,
		y=trace_min_time,
		col sep=semicolon
	]
	{data/RandomListGenerator/AVL.csv};
\addlegendentry{AVL}

\addplot [smooth, color=Dandelion, mark=*, mark size=0.5pt]
	plot [error bars/.cd, y dir = both, y explicit]
	table[
		x=size,
		y=find_min_time,
		col sep=semicolon
	]
	{data/RandomListGenerator/BST.csv};
\addlegendentry{BST}
\end{axis}
\end{tikzpicture}
\medskip
\subsection{Czas Przejścia Drzewa In Order}
\medskip
\begin{tikzpicture}
\begin{axis}[
	width=10cm,
	height=5cm,
	ylabel=$time$,
	xlabel=$size$,
	y unit=s, y SI prefix=nano,
	title={Trace In Order Time}
	]
\addplot [smooth, mark=*, mark size=0.5pt]
	plot [error bars/.cd, y dir = both, y explicit]
	table[
		x=size,
		y=trace_in_order,
		col sep=semicolon
	]
	{data/RandomListGenerator/AVL.csv};
\addlegendentry{AVL}

\addplot [smooth, color=Dandelion, mark=*, mark size=0.5pt]
	plot [error bars/.cd, y dir = both, y explicit]
	table[
		x=size,
		y=trace_inorder_time,
		col sep=semicolon
	]
	{data/RandomListGenerator/BST.csv};
\addlegendentry{BST}
\end{axis}
\end{tikzpicture}
\medskip
\subsection{Czas Balansowania Drzewa}
\medskip
\begin{tikzpicture}
\begin{axis}[
	width=10cm,
	height=5cm,
	ylabel=$time$,
	xlabel=$size$,
	y unit=s, y SI prefix=nano,
	title={Balance BST}
	]
	
\addplot [smooth, color=Dandelion, mark=*, mark size=0.5pt]
	plot [error bars/.cd, y dir = both, y explicit]
	table[
		x=size,
		y=balance_time,
		col sep=semicolon
	]
	{data/RandomListGenerator/BST.csv};
\end{axis}
\end{tikzpicture}
\medskip
\section{Podsumowanie}


Na podanych wykresach widać, różnice między AVL i BST. Przy tworzeniu BST dokonujemy prostych porównań i wsadzamy w pierwsze wolne miejsce nowy node. Inaczej sprawa wygląda w AVL gdzie po każdorazowym wsadzeniu danych musimy przebudować drzewo, równoważąc je. Istnieje jednak wyjątek od takich sytacji, jeśli dane dodawane do drzewa AVL są, pierwotnie posegregowane. W takiej sytuacji nie trzeba drzewa równoważyć, bo dane układają się same.

Ten minus AVL powinien być jednak mniej dotkliwy w perspektywie częstego wyszukiwania w drzewia danych. AVL z definicji zapewnia średnią najkrótszą drogę dostępu do nodów.

Czemu więc wyszukiwanie najmniejszej wartości pokrywa się czasowo z BST? Trudno na to odpowiedzieć. Trzeba wziąć pod uwagę, że implementacja tych struktur wraz z testami odbyła się przy pomocy Javy. Duże wahania mogą zostać tym faktem wytłumaczone, tym bardziej, że czasy działania algorytmów są niesamowicie małe.

Próbkowanie odbyło się na grupach o liczności 30 egzemplarzy na każdy krok, co w teorii powinno zminimalizować różnice i wypłaszczyć wykresy. Niestety przez możliwe odchyły maksymalne, średnia zamiast trzymać się "idealnego wykresu" zaczyna pokazywać coraz więcej szumów. Do podobnej sytuacji doszło przy zmniejszeniu kroku generowanych wartości.

Zaobserwować mnożna, że balansowanie, drzewa jest niemalże liniowe, mimo widocznych szumów.

\section{Kod}

\href{https://github.com/McSymilian/AISD_PUT_Testing_Support/tree/master/src/main/java/org/data_structures/tree}{GitHub repository}
\end{document}